\documentclass{article}
\begin{document}
%--------------------------------------------------------------------------
\section{initial questions}
%--------------------------------------------------------------------------
why is a-sio2 fractal dimension 2.7? how do you measure fractal dimension 
of a structure? this will help figure out how to measure it for an 
atomistic structure.

what does a-sio2 accumulation look like?

%--------------------------------------------------------------------------
\section{paper comments}
%--------------------------------------------------------------------------
"We apply a simple-cubic sine-type dispersion to confined acoustic 
vibrations to approximate vibrational dispersion due to the tetrahedral 
orientation of bonds in certain dielectric materials."

First, I'm not sure you need to assume a simple-cubic crystal type to 
use a sine-type dispersion. For example, a sine-type dispersion describes 
a 1-D monatomic chain. I would just say simple sine-type, or something 
like that.

Why is a simple-cubic sine-type dispersion used "due to" the 
tetrahedrally-coordinated materials? I think  that the psuedo dispersion 
of amorphous materials has been shown to follow a simple sine-type, at 
least at low-freq/low-wavenumber.  Examples are a-Si and a-SiO2. The 
pseudo-dispersion can be more complicated at large wavenumber, 
as you can see from Fig. 3 of my amorphous draft.


%--------------------------------------------------------------------------
\section{ongoing discussion}
%--------------------------------------------------------------------------

%--------------------------------------------------------------------------
\section{buchneau soft potential model}
%--------------------------------------------------------------------------
\cite{buchneau_interaction_1992}

%--------------------------------------------------------------------------
\section{orbach fracton model}
%--------------------------------------------------------------------------

Based on your reference to the fracton model, 
I am reading \cite{orbach_phonon_1993}:

"That the introduction of vibrational
anharmonicity allows for thermal trans-
port is an interesting feature of random
structures. Whereas in ordered struc-
tures, anharmonicity serves to reduce
heat flow, in disordered structures anhar-
monicity is the cause of heat flow."

This is an interesting comment about anharmponicity that deserves 
alot of attention. 

Allen-Feldman claim that harmonic scattering is adequate to explain the 
features of T(K) for thin film a-Si which does NOT show the plateau in 
T(K) < 10 K:
\cite{feldman_numerical_1999}:

"The intrinsic broadening due to disorder is strong enough to suppress
thermal conductivity to the level seen experimentally, with no need for 
special anharmonic effects or localization, except for the influence of 
two-level systems on the modes at very low frequencies."

Here is their conclusion after using a diffusivitiy scaling 
$D \propto \omega^{-2}$:

"The conclusion is that intrinsic harmonic glassy disorder
contained in our finite calculation kills off the heat-carrying
ability of propagons rapidly enough without invoking any
exotic mechanism. Our ␬ (T) curve is reminiscent of the ex-
periments of Zaitlin and Anderson after holes are introduced
to enhance the elastic damping of long-wavelength modes.
The plateau disappears from their data in much the same way
that it disappears from our theory due to extra damping of
small-Q propagons."

They explain T(K) for a-Si which DOES show the plateau by using a 
Rayleigh scaling $D \propto \omega^{-4}$:\cite{feldman_thermal_1993}

"The harmonic diffusivity becomes a Rayleigh m
law
and gives a divergent ~(T) as T~O. To eliminate this we make the standard 
assumption of resonant-plus-relaxational absorption from two-level systems 
(this is an anharmonic effect which would lie outside
our model even if it did contain two-level systems implicitly)."

The low-freq scaling of the diffusivities Depending on whether 
u see the plateau in T(K), at least according to AF. 

The effect of anharmonicity is introduced by AF when using 
the Rayleigh type scaling, so it is hard 
to judge whether what the fracton model says is true.  The AF theory 
seems to be able to describe both situations for a-Si (plateau and 
no plateua). This will be an important 
topic when you bring it up in the group meeting presentation.

orbach:

"Experimentally, Eq. (2)accounts for
the increase in n(T) for temperatures
above the plateau for many amorphous
and glassy materials (see Fig. 1)."

This is explained by AF theory as a specific heat effect of the mid-range 
and high-freq modes.  Again, these two theories are explaining the 
same feature in 2 different ways, so you should focus on these features. 

%--------------------------------------------------------------------------


\end{document}
